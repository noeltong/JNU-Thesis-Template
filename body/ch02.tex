% !Mode:: "TeX:UTF-8"
% !TeX encoding = UTF-8 Unicode
% !TeX program = xelatex
% !TeX root = ../root.tex


\chapter{文档中一些格式的实现方式}
\label{ch02}

\section{字体与字号}

\LaTeX{}提供了多种字体与字号的设置方式。字号大小在本模板中已经预先定义。

\begin{table}[h]
    \centering
    \caption{不同类型的文字形式}
    \label{table1}
    \begin{tabular}{cc}
    \toprule
    命令 & 效果  \\
    \midrule
    \texttt{\textbackslash textbf\{江南大学\}}  & \textbf{江南大学} \\
    \texttt{\textbackslash texttt\{江南大学\}}  & \texttt{江南大学} \\
    \bottomrule
    \end{tabular}
\end{table}


\section{浮动体}

\subsection{表格}

在江南大学本科生毕业设计(论文)模板中,表格要求为三线表,且表格标题位于表格主体的上方,相对于表格居中对齐。

在\LaTeX{}中,表格的实现需要在\texttt{table}环境中,以\texttt{\&}表示分列,以\texttt{\textbackslash\textbackslash}表示换行。例如

\begin{lstlisting}[language=tex, breaklines=true, basicstyle=\ttfamily, numbers=left, numberstyle=\tiny, frame=shadowbox]
    \begin{table}[h]
        \centering
        \caption{这是一个表格示例}
        \label{table1}
        \begin{tabular}{cccc}
        \toprule
        第一列 & 第二列 & 第三列 & 第四列 \\
        \midrule
        1   & 2   & 3   & 4   \\
        5   & 6   & 7   & 8   \\
        \bottomrule
        \end{tabular}
    \end{table}
\end{lstlisting}

在文档的相应位置插入以上代码,得到的结果即为

\begin{table}[h]
    \centering
    \caption{这是一个表格示例}
    \label{table1}
    \begin{tabular}{cccc}
    \toprule
    第一列 & 第二列 & 第三列 & 第四列 \\
    \midrule
    1   & 2   & 3   & 4   \\
    5   & 6   & 7   & 8   \\
    \bottomrule
    \end{tabular}
\end{table}

如果表格行距过大或过小,可以通过\texttt{spacing}宏包设置。

\begin{lstlisting}[language=tex, breaklines=true, basicstyle=\ttfamily, numbers=left, numberstyle=\tiny, frame=shadowbox]
    \begin{table}[h]
        \centering
        \caption{这是一个表格示例}
        \label{table1}
        \begin{spacing}{0.5}
        \begin{tabular}{cccc}
        \toprule
        第一列 & 第二列 & 第三列 & 第四列 \\
        \midrule
        1   & 2   & 3   & 4   \\
        5   & 6   & 7   & 8   \\
        \bottomrule
        \end{tabular}
        \end{spacing}
    \end{table}
\end{lstlisting}

\begin{table}[h]
    \centering
    \caption{这是一个表格示例}
    \label{table1}
    \begin{spacing}{0.5}
    \begin{tabular}{cccc}
    \toprule
    第一列 & 第二列 & 第三列 & 第四列 \\
    \midrule
    1   & 2   & 3   & 4   \\
    5   & 6   & 7   & 8   \\
    \bottomrule
    \end{tabular}
    \end{spacing}
\end{table}

\subsection{图片}

图片的实现方式与表格类似。如下所示为一个最简单的图片插入方式,包含图片的居中、插入以及设置图片的标题。

\begin{lstlisting}[language=tex, breaklines=true, basicstyle=\ttfamily, numbers=left, numberstyle=\tiny, frame=shadowbox]
    \begin{figure}[h]
        \centering
        \includegraphics[width=0.5\textwidth]{LOGO.png}
        \caption{江南大学校标}
    \end{figure}
\end{lstlisting}

其效果为

\begin{figure}[h]
    \centering
    \includegraphics[width=0.5\textwidth]{LOGO.png}
    \caption{江南大学校标}
\end{figure}

其中,\texttt{\textbackslash centering}表示将图片居中,在\texttt{\textbackslash includegraphics[width=0.5\textbackslash textwidth]\{LOGO.png\}}中,前一个方括号中的内容表示图片的宽度,之后花括号中的内容为图片的完整文件名。在本模板中,预设了图片的目录为项目根目录中的\texttt{figures}文件夹。

此外,还可以设置多个文件的形式,需要使用到\texttt{subfloat}宏包,此宏包已经在文档中预先插入。

\begin{lstlisting}[language=tex, breaklines=true, basicstyle=\ttfamily, numbers=left, numberstyle=\tiny, frame=shadowbox]
    \begin{figure}
        \centering
        \subfloat[Title 1]{
            \includegraphics[width=0.4\textwidth]{LOGO.png}
        }
        \subfloat[Title 2]{
            \includegraphics[width=0.4\textwidth]{LOGO.png}
        } \\
        \subfloat[Title 3]{
            \includegraphics[width=0.4\textwidth]{LOGO.png}
        }
        \subfloat[Title 4]{
            \includegraphics[width=0.4\textwidth]{LOGO.png}
        }
        \caption{多个图片的插入方式效果预览}
    \end{figure}
\end{lstlisting}

第八行的换行符号是为了让四张图片在第二张处换行,避免长度过长超出文档宽度。其效果如图\ref{sty2}。

\begin{figure}[h]
    \centering
    \subfloat[Title 1]{
        \includegraphics[width=0.4\textwidth]{LOGO.png}
    }
    \subfloat[Title 2]{
        \includegraphics[width=0.4\textwidth]{LOGO.png}
    } \\
    \subfloat[Title 3]{
        \includegraphics[width=0.4\textwidth]{LOGO.png}
    }
    \subfloat[Title 4]{
        \includegraphics[width=0.4\textwidth]{LOGO.png}
    }
    \caption{多个图片的插入方式效果预览}
    \label{sty2}
\end{figure}

\section{参考文献}

在本模板中,使用\BibTeX{}管理参考文献。其文件位于根目录下的\texttt{references.bib},参考文献的样式文件为\texttt{jn.bst}。其中,参考文献的样式不需要做更改。在论文撰写过程中。需要将参考文献的\BibTeX{}文献记录添加在\texttt{references.bib}中,在文章中进行引用,然后使用\XeLaTeX{}进行编译。

文献记录不必一行一行的亲自码掉。目前在各大主流论文数据库和检索网站都提供\BibTeX{}格式的文献记录的下载。对于中文文献,可以使用中国知网\footnote{中国知网,\url{https://www.cnki.net/}}进行查询;对于英文文献,可以使用Google Scholar\footnote{Google Scholar, \url{https://scholar.google.com},需要科学上网}或Microsoft Academic\footnote{Microsoft Academic, \url{https://academic.microsoft.com/home}}进行查询。

\begin{lstlisting}[language=tex, breaklines=true, basicstyle=\ttfamily, numbers=left, numberstyle=\tiny, frame=shadowbox]
    @inproceedings{vaswani2017attention,
	title="Attention is All you Need",
	author="Ashish {Vaswani} and Noam {Shazeer} and Niki {Parmar} and Jakob {Uszkoreit} and Llion {Jones} and Aidan N. {Gomez} and Lukasz {Kaiser} and Illia {Polosukhin}",
	booktitle="Proceedings of the 31st International Conference on Neural Information Processing Systems",
	volume="30",
	pages="5998--6008",
	year="2017"
    }
\end{lstlisting}


上面就是一个\BibTeX{}样例。以\texttt{\@}符号开始,以\texttt{\{\}}包括所有的内容,在\texttt{\{}之后紧随的是每个文献记录的key。接下来\texttt{=}之前的都是固定的条目,\texttt{=}后面的是具体的内容。\BibTeX{}靠着key识别每个记录,这样在引用的时候就不会出错了。

值得一提的是,在\BibTeX{}中,文献的入口不能重复,这意味着相同的文献,其记录不能出现两次,即同一文献的记录不能重复添加。这一错误在编译时不会体现在\XeLaTeX{}与\BibTeX{}的错误信息中,需要大家在编译输出中自行作出判断。

在引用时,需要用到\texttt{cite}命令。

\begin{lstlisting}[language=tex, breaklines=true, basicstyle=\ttfamily, numbers=left, numberstyle=\tiny, frame=shadowbox]
    Attention的原理就是计算当前输入序列与输出向量的匹配程度,匹配度高也就是注意力集中点其相对的得分越高,其中Attention计算得到的匹配度权重,只限于当前序列对,不是像网络模型权重这样的整体权重\cite{vaswani2017attention}。
\end{lstlisting}

其效果为:

\noindent Attention的原理就是计算当前输入序列与输出向量的匹配程度,匹配度高也就是注意力集中点其相对的得分越高,其中Attention计算得到的匹配度权重,只限于当前序列对,不是像网络模型权重这样的整体权重\cite{vaswani2017attention}。

编译输出的参考文献样式可以参考文末的参考文献部分。

\section{引用}

在文章写作过程中,有时候需要对文章其他部分的内容(引用其他章节,或引用文档其他位置的图片或表格,或注明图片或表格的位置),此时需要使用引用。一个简单的例子为

江南大学标志如图\ref{sty1}所示。

\begin{figure}[h]
    \centering
    \includegraphics[width=0.5\textwidth]{LOGO.png}
    \caption{江南大学校标}
    \label{sty1}
\end{figure}

\begin{lstlisting}[language=tex, breaklines=true, basicstyle=\ttfamily, numbers=left, numberstyle=\tiny, frame=shadowbox]
    江南大学标志如图\ref{sty1}所示。

    \begin{figure}[h]
        \centering
        \includegraphics[width=0.5\textwidth]{LOGO.png}
        \caption{江南大学校标}
        \label{sty1}
    \end{figure}    
\end{lstlisting}