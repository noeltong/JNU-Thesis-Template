% !Mode:: "TeX:UTF-8"
% !TeX encoding = UTF-8 Unicode
% !TeX program = xelatex
% !TeX root = ../root.tex


\chapter{环境与编译}
\label{ch03}

\section{概述}

本章将会对模板使用的具体步骤进行描述。

\section{预备工作}

\begin{enumerate}
    \item 对\texttt{setup/settings.tex}中文章的标题和作者进行设置。
    \item 修改\texttt{cover.doc},并生成\texttt{cover.pdf},替换原有的\texttt{cover.pdf}。
    \item 修改(如有必要)\texttt{setup/userdefs.tex}添加用户自定义符号或命令。
\end{enumerate}

\section{撰写正文}

模板已经将文档划分为了若干文件,增加了编译速度。

首先修改\texttt{preface}文件夹中的两个文件,得到中文摘要和英文摘要,并修改文末的关键词。此时根据设计的类型(论文或设计),修改\texttt{jnthesis.cls}文件中第64行与第153行的“设计总说明”为对应的类型(“摘\~要”或“设计总说明”)。

其次,修改\texttt{body}文件夹下的若干文件,并根据章节的编排在\texttt{main.tex}里增加对应行来加入对应的文件或者删除或注释掉无用的行。同时,根据文章的参考文献,在\texttt{references.bib}中添加对应的信息,并在正文中相应位置\texttt{cite}对应的key。

在正文撰写完成之后,修改\texttt{appendix}文件夹下的两个文件,分别为致谢以及发表的论文列表。若没有论文成果发表,需要在\texttt{root.tex}中注释掉第28行。

\section{编译}

由于模板使用了\BibTeX{},包含了引用,完整的编译过程为:

\begin{enumerate}
    \item \XeLaTeX{}
    \item \XeLaTeX{}
    \item \BibTeX{}
    \item \XeLaTeX{}
\end{enumerate}

如果更改了章节编排,包括新增章节、修改章节顺序或删除章节,以及增加了引用等,正确的\texttt{root.pdf}文件会在执行两次\XeLaTeX{}编译之后产生。

如果修改了参考文献,则需要完整执行一次上述的四次编译才可以得到正确的\texttt{root.pdf}文件。